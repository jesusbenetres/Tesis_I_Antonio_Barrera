\chapter{PLANTEAMIENTO DEL PROBLEMA}
\section{Descripción de la Realidad Problemática}

En la actualidad, el uso de tecnologías avanzadas, como la inteligencia artificial (IA), ha permitido la creación de contenido multimedia falsificado conocido como deepfake. Este contenido incluye videos, imágenes y, más recientemente, audios. Los deepfakes de audio han adquirido una relevancia particular debido a su capacidad para imitar voces humanas con una precisión asombrosa, lo que ha desencadenado una serie de problemas relacionados con la seguridad y la confianza en las comunicaciones digitales (Heidari, 2023). Esta tecnología ha comenzado a ser utilizada para fraudes y suplantación de identidad, presentando un desafío significativo para la detección de audios manipulados, especialmente en español, donde las herramientas actuales de detección no están adecuadamente adaptadas para capturar las particularidades del idioma.

En Perú, el fraude mediante deepfakes de audio ha mostrado un aumento considerable en los últimos años. Un informe de la Policía Nacional del Perú (PNP) revela que más de un millón de soles han sido robados mediante la clonación de voces utilizando IA. En estos fraudes, los delincuentes suelen imitar la voz de familiares o amigos para solicitar dinero en situaciones de emergencia falsas, lo que resulta en un engaño muy difícil de detectar por las víctimas (Rojas Berríos, 2023). Este incremento de fraudes plantea un problema general relacionado con la incapacidad de las herramientas actuales para detectar audios falsificados, lo que facilita la comisión de delitos en el ámbito digital.

Uno de los problemas específicos más críticos es la falta de un dataset en español que contemple variaciones regionales y voces manipuladas para entrenar modelos de redes neuronales profundas. La mayoría de los avances en la detección de deepfakes de audio se han centrado en el inglés, lo que genera una brecha importante en la capacidad de los modelos para adaptarse a las características del habla en español, que presenta variaciones significativas en términos de patrones de voz, frecuencia fundamental y ritmo del habla (Heidari, 2023). Esta falta de datos específicos en español dificulta la construcción de modelos robustos que puedan detectar deepfakes con precisión en este idioma.

Otro problema específico está relacionado con las limitaciones técnicas de las herramientas actuales para analizar variables clave como el tono de voz, timbre de voz y formantes. Los audios falsificados mediante deepfakes logran imitar estas características acústicas de manera casi perfecta, lo que confunde tanto a los sistemas de detección como a las personas. En particular, los deepfakes en español plantean desafíos únicos debido a las diferencias fonéticas con otros idiomas, lo que complica aún más la detección efectiva de manipulaciones (Amesquita Cuya, 2023).

Un tercer problema específico se refiere al uso de deepfakes de audio en fraudes por suplantación de identidad en Perú, donde los delincuentes utilizan esta tecnología para hacer pasar sus voces por las de familiares o colegas en situaciones de emergencia o negociación. Estos fraudes, que afectan tanto a individuos como a empresas, se ven agravados por la dificultad de las técnicas actuales para analizar características como la prosodia, articulación, transiciones entre fonemas y ruidos de fondo. Estos factores son críticos para la identificación precisa de audios falsificados, ya que los deepfakes no siempre logran replicar con fidelidad estos aspectos del habla humana, pero los sistemas de detección existentes no están optimizados para capturarlos (Chen \& Magramo, 2024).

A nivel global, se ha documentado un incremento en el uso de deepfakes para cometer fraudes, con pérdidas millonarias en varios países. En Hong Kong, por ejemplo, se reportó un caso en el que un trabajador fue engañado mediante una videollamada con deepfakes de varios miembros de la junta directiva de su empresa, lo que resultó en un fraude de 25 millones de dólares (Chen \& Magramo, 2024). Este tipo de casos destaca la urgencia de desarrollar soluciones tecnológicas más efectivas que permitan la detección de deepfakes de audio en español, donde las herramientas actuales siguen siendo insuficientes.

En el contexto peruano, la proliferación de fraudes basados en deepfakes de audio no solo afecta a los individuos, sino también a figuras públicas. Un caso reciente involucró a la presidenta Dina Boluarte, cuya voz fue manipulada mediante IA para hacer parecer que promovía una inversión fraudulenta, lo que generó confusión entre el público y demostró el poder de esta tecnología para influir en la opinión pública y causar daño reputacional (Amesquita Cuya, 2023).

Por todo lo anterior, es evidente que la detección de deepfakes de audio en español requiere un enfoque más sofisticado. El desarrollo de modelos basados en redes neuronales profundas que puedan analizar variables acústicas clave, como el tono de voz, timbre de voz, patrones de voz, frecuencia fundamental, duración y ritmo del habla, formantes, nivel de energía del habla, ruidos de fondo, prosodia, articulación y transiciones entre fonemas, es esencial para mitigar el impacto de los fraudes por deepfakes en Perú. Estas variables juegan un papel fundamental en la autenticidad del habla y pueden proporcionar pistas valiosas para identificar audios manipulados. Sin embargo, la falta de modelos especializados en español y la escasez de datasets específicos siguen siendo los principales obstáculos para lograr una detección precisa y confiable.

\section{Formulación del Problema}

\subsection{Problema General}
\newcommand{\ProblemaGeneral}{
	El incremento del fraude en Perú mediante el uso de tecnologías deepfake de audio ha evidenciado la falta de herramientas adecuadas para detectar estos fraudes, especialmente en español. Las técnicas actuales no logran identificar eficazmente las características acústicas del español, como el tono de voz, timbre de voz, patrones de voz, frecuencia fundamental (pitch), duración y ritmo del habla, formantes, nivel de energía del habla (intensidad), ruidos de fondo, prosodia, articulación y transiciones entre fonemas, lo que facilita la suplantación de identidad y el fraude en las comunicaciones personales y empresariales. 
}
\ProblemaGeneral
\subsection{Problemas Espec\'{i}ficos}
\newcommand{\Pbone}{
La falta de un dataset en español que incluya variaciones regionales y voces manipuladas dificulta el entrenamiento de modelos de redes neuronales profundas para detectar deepfakes de audio en español, debido a las diferencias en patrones de voz, frecuencia fundamental (pitch) y ritmo del habla.
}
\newcommand{\Pbtwo}{
Las técnicas actuales no logran detectar las variaciones en el tono de voz, timbre de voz y formantes en audios en español, lo que disminuye la precisión en la identificación de audios manipulados.
}
\newcommand{\Pbthree}{
Los fraudes por suplantación de identidad mediante deepfakes de audio en Perú son difíciles de detectar con las técnicas actuales debido a la falta de análisis de prosodia, articulación, transiciones entre fonemas y ruidos de fondo, lo que incrementa el riesgo de fraude.
}


\begin{itemize}
	\item \Pbone
	\item \Pbtwo
	\item \Pbthree

\end{itemize}

\section{Objetivos de la Investigación}
Para la formulación de los objetivos de la presente investigación se elaboró un «árbol de objetivos» (véase Anexo 2) 
\subsection{Objetivo General}
\newcommand{\ObjetivoGeneral}{
Desarrollar un modelo basado en redes neuronales profundas que permita detectar deepfakes de audio en español mediante el análisis de variables clave como tono de voz, timbre de voz, patrones de voz, frecuencia fundamental, duración y ritmo del habla, formantes, nivel de energía del habla, ruidos de fondo, prosodia, articulación y transiciones entre fonemas, mejorando la precisión en la identificación de audios manipulados para mitigar fraudes por suplantación de identidad en Perú.
}
\ObjetivoGeneral
\subsection{Objetivos Espec\'{i}ficos}
\newcommand{\Objone}{
Desarrollar un dataset específico en español, con variaciones regionales y voces manipuladas, para entrenar un modelo de redes neuronales profundas que detecte deepfakes de audio
}
\newcommand{\Objtwo}{
Implementar un modelo de redes neuronales profundas que analice el tono de voz, timbre de voz y formantes para mejorar la precisión en la detección de deepfakes de audio en español.
}
\newcommand{\Objthree}{
Evaluar la eficacia del modelo de redes neuronales profundas en la detección de deepfakes de audio en contextos de fraude por suplantación de identidad en Perú, considerando prosodia, articulación, transiciones entre fonemas y ruidos de fondo.
}

\begin{itemize}
	\item {\Objone}
	\item {\Objtwo}
	\item {\Objthree}
\end{itemize}

\section{Justificación de la Investigación}

\subsection{Teórica}
Esta investigación se realiza 

\subsection{Práctica}
Al culminar la investigación 

\subsection{Metodológica}. 

\section{Delimitación del Estudio}

\subsection{Espacial}
Para la presente investigación 

\subsection{Temporal}
Los datos que serán necesari. 

\subsection{Conceptual}
Esta investigación se 

\section{Hipótesis}

\subsection{Hipótesis General}
\newcommand{\HipotesisGeneral}{
El uso de un modelo basado en redes neuronales profundas que analice las variables acústicas clave como tono de voz, timbre de voz, patrones de voz, frecuencia fundamental, duración y ritmo del habla, formantes, nivel de energía del habla, ruidos de fondo, prosodia, articulación y transiciones entre fonemas mejora significativamente la precisión en la detección de deepfakes de audio en español, reduciendo el riesgo de fraudes por suplantación de identidad en Perú.
}
\HipotesisGeneral
\subsection{Hipótesis Específicas}
\newcommand{\Hone}{
La creación de un dataset en español que incluya variaciones regionales y voces manipuladas mejorará significativamente la capacidad de las redes neuronales profundas para detectar deepfakes de audio en este idioma
}
\newcommand{\Htwo}{
El análisis del tono de voz, timbre de voz y formantes mediante redes neuronales profundas aumentará la precisión en la detección de deepfakes de audio en español.
}
\newcommand{\Hthree}{
El modelo de redes neuronales profundas será más efectivo en la detección de deepfakes en contextos de fraude en Perú al incluir el análisis de prosodia, articulación, transiciones entre fonemas y ruidos de fondo, en comparación con las técnicas actuales.	
}

\begin{itemize}
	\item \Hone
	\item \Htwo
	\item \Hthree
\end{itemize}

\subsection{Matriz de Consistencia}
A continuación se presenta la matriz de consistencia elaborada para la presente investigación (véase Anexo \ref{1:table}).

